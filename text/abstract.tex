Social distancing interventions can be effective against epidemics but are potentially detrimental for the economy.
Businesses that rely heavily on face-to-face communication or close physical proximity when producing a product or providing a service are particularly vulnerable. 
There is, however, no systematic evidence about the role of human interactions across different lines of business and about which will be the most limited by social distancing.
Here we provide theory-based measures of the reliance of U.S. businesses on human interaction, detailed by industry and geographic location.
We find that, before the pandemic hit, 35 million workers worked in occupations that rely heavily on face-to-face communication or require close physical proximity to other workers. Many of these workers lost their jobs since. Consistently with our model, employment losses have been largest in sectors that rely heavily on customer contact and where these contacts dropped the most: retail, hotels and restaurants, arts and entertainment and schools. Our results can help quantify the economic costs of social distancing.
